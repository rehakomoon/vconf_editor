\RequirePackage{plautopatch} % pLaTeX非互換のパッケージへのパッチを自動適用
\RequirePackage{fix-cm} % cm フォントのサイズが 10pt でエラーになるのを修正
\documentclass[10pt,a4paper,autodetect-engine,dvipdfmx]{jsarticle} % upLaTeX と pLaTeX2e の判別を自動化, dvipdfmxをグローバルに指定
\usepackage{graphicx} % dvipdfmxはクラスオプションに移動
\usepackage{url}
\usepackage{here}
\usepackage{vconf2023}

\title{バーチャル学会2023要旨テンプレート(タイトルを記入)}
\author{はこつき (Twitter: @re\_hako\_moon)$^\star$(著者を記入.連絡先を忘れずに記載すること)}

\begin{document}
\begin{abstract}
ここに概要を記入する.
テンプレートではこの概要欄を利用して,バーチャル学会の投稿時にとくに注意する事項を記載する.
1) 著者情報に連絡先を記載すること.
2) Webサイトを引用する場合には,ページタイトル,URL,参照日を記載すること.
3) 著者情報に所属を含める場合,それぞれの所属がわかるように記載すること.
不明な点がある場合にはバーチャル学会実行委員会までお気軽にご連絡ください.
\end{abstract}

\maketitle

\begin{figure}[h]
\centering
\includegraphics[width=0.9\linewidth]{vconf2023.png}
\caption{ティザー画像を表示する場合には,ここに図として挿入してもよい.}
\label{fig:topfigure}
\end{figure}

\begin{multicols}{2}

\section{緒言}

本テンプレートはLaTeXを用いてバーチャル学会の要旨原稿を作成するためのファイル群である.

\section{原稿の書式}

\subsection{全体のフォーマット}

A4判用紙のPDFで提出する.WordもしくはLaTeXファイルでの作成を原則とし,レイアウトは2カラム,フォントサイズはタイトル18pt・本文10ptとする.最大ページ数は4ページとし,フォントは和文は明朝体,英文はTimes New Roman,Centuryなどの標準的なものを使用する.

\begin{itemize}
\item タイトルページ:論文の1ページ目に標題・著者名・所属・要旨を記載する.複数著者で場合を記載する場合,それぞれの所属がわかるように記載するよう注意する.
\item 標題:意味のない語の使用を避けて具体的に記述する.
\item 著者名・所属:著者名にはハンドルネームを使用することも可能であり,代表者(連絡先)に*をつける.所属は任意に記載し複数の場合は1. 2…と記す.
\item 概要:概要は単独で理解可能とし,1/2頁以内とする.
\end{itemize}

\subsection{内容について}

発表要旨は次の事項を基本として,簡潔に記載することが望ましい.

\begin{itemize}
\item 緒言:研究の系譜と研究目的を明示し,総説的記述は避ける.
\item 研究方法:簡単明瞭にまとめる.既発表の研究方法はその文献のみを示し,詳細は避ける.
\item 結果:簡単明瞭にまとめる.研究結果および考察をまとめてもよい.
\item 考察:論旨の簡明を旨とし,著者の研究結果の意義の説明に限定し,論拠のない仮説は避ける.
\item 参考文献:本文中に引用順に列記する.
\end{itemize}

\subsection{表記方法}

本文,大見出し,小見出しなどを明瞭にすること.図および表については原稿中の指定位置と幾分異なることもあるため,本文中では「つぎの表」のような表現を避け, Fig. 1 または Table 1 のように書く.また,図表のキャプションはそれ単体で読んだ際に図表の意味が理解できるように簡潔に記述する.

\subsection{引用文献}

本文中の引用順序に著者名または事項の後ろに\cite{rafferty1994},\cite{vconf2023}のような符号(\cite{okatani2015,kataoka2016}のようにまとめてもよい)をつけ,文末に番号順に列記する.引用文献が2名までの共著者はその全員を本文中に示し,3名以上の場合はまとめること,また論文末の文献欄では全著者名を列記することが望ましい.Webサイトを引用する場合,ページタイトル,URL,参照日を記載する.

\subsection{字体・記号略号}

下記の点に注意し,判読・理解しやすい原稿となるよう努める.

\begin{itemize}
\item 上つき文字,下つき文字:小さく間違いやすいから特に注意し,その位置を明確に示す.
\item 学名:生物種の学名はイタリック体とする.
\item 略語:略語を用いる場合には,初出時に正式名称を表記(スペルアウト)する.2回目からは訳語で表記する.
\item 量単位:国際単位系(SI)を用いる.
\item 数式:独立した式の文字(変数)は指定のない限りイタリック体に組まれる.式の一連番号を( )で囲み,頁の右端に書く.文中の式は一行に収める.
\item 脚注:本文中に*, a, bなどを右肩につけ,そのページの下に横線を引き,下に記述する.
\end{itemize}

\subsection{図表の作成}

図および表の説明は,本文を見なくても大要が把握できる程度の最小限のものであることが望ましい.これらは図\ref{fig:sample_figure}のようにすべて本文中から参照する.

図については題名・説明文ともに下に記述する.写真も図として扱う.顕微鏡写真などの縮小,拡大を正確に示すべき図では,必ず図中に標準尺度を示す線を記入する(×1000等では示さない).

表は肝要にして記述に値するデータのみを表とし,題名は上に,説明文は下に記述する.表の題名はその終わりにピリオドを入れる.表中の列の頭には適切な題をつけ,適切な略字を用いて短くする.単位を明示する.各行・各列列に番号をつけることは,本文引用に必要な場合を除き避ける.

\begin{figure}[H]
\centering
\includegraphics[width=0.95\linewidth]{vconf2023.png}
\caption{図のキャプションの例.}
\label{fig:sample_figure}
\end{figure}

\section*{謝辞}

バーチャル学会2023 LaTeX版テンプレートの作成にあたり,きゅーしす様 (Twitter: @Queue\_sys) にご助言・ご助力いただきました.心より深く感謝申し上げます.

\bibliographystyle{plain}
\begin{thebibliography}{9}
\bibitem{rafferty1994} W. Rafferty, "Ground antennas in NASA’s deep space telecommunications," Proc. IEEE vol. 82, pp. 636-640, May 1994.
\bibitem{vconf2023} バーチャル学会実行委員会, "バーチャル学会2023 Webサイト." \url{https://vconf.org/2023/} (参照 2023-06-30).
\bibitem{okatani2015} 岡谷貴之, "深層学習," 2015.
\bibitem{kataoka2016} Yun He, et al. "Human Action Recognition without Human," In proceedings of the ECCV Workshop, 2016.
\end{thebibliography}

\end{multicols}
\end{document}
